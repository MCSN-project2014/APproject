
\chapter{\label{chapter5} The \fsharp Code Generator}

The code generator module is responsible for translating the \fwap intermediate representation (AST) into compilable \fsharp programs. The module itself can be thought of as a fully working \textit{compiler} (i.e. a computer program performing a translation from a source into another target language, \cite{dragon}). The executable for this piece of software is contained within the release file \textit{funwapc.exe}. 

During this chapter, we will go through the organization of the \texttt{FSCodeGen.cs} class, in which the whole translator has been implemented.

\section{The \texttt{FSCodeGen.cs} class}

The overall design of the compiler could be intuitively explained by having a look at the method \texttt{translate (ASTNode n)}, partially listed below. \\

\lstinputlisting[firstline=70,lastline=118,caption={The \texttt{translate(ASTNode n)} method.}]{../APproject/FSCodeGenerator/FSCodeGen.cs }

Depending on the node's label, \texttt{translate(ASTNode n)} calls the appropriate method to perform the translation and it is recursively called by the auxiliary method \texttt{translateRecursive (ASTNode n)}, which also prints the terminal nodes when needed (i.e. the AST leaves). The aforementioned process performs a Breadth-First search over the AST structure and writes the \fsharp code into a specified output file (\textit{a.fs} by default).

While a good amount of the translating methods are trivial to understand for any programmer, a more detailed treatment is reserved to the assignments translation methods, \texttt{translateAssig(ASTNode n)} and \texttt{translateAssigDecl(ASTNode n)}, which also compiles the \textbf{async\{\}} and \textbf{dasync\{\}} constructs.

\section{Translating the Assignments}

The assignment translation -- and, analogously, the declaration/assignment -- was subject to a subtle work in order to manage the anonymous functions and the \textbf{async\{\}} and \textbf{dasync\{\}} conversion into \fsharp. 

Firstly, using the \texttt{ref} cell variables everywhere in the target code was an inducted design choice, since \fsharp anonymous functions cannot modify external \texttt{mutable} variables. Even if \texttt{mutable}'s were the team's first choice, being lighter for the target run-time support, the mixed use with \texttt{ref} was not worth the implementation effort. Also, it turned out to be fairly simple to add the \textbf{!} (\textit{bang} operator) when it was needed on the right handside of some assignment statement.

Secondarily, the same class \texttt{Environment.cs} used for the interpreter (see Chapter~\ref{chapter4}) has been adopted here in order to keep track of the association between each variable and a boolean value, set to true when that given variable is object of an async/dasync operation.The boolean value is then reset to false, immediately before issuing another non-async/non-dsync assignment of the variable. Asynchronous calls are translate into .NET \texttt{Task}'s, as shown below.

\begin{lstlisting}[caption=Example of \texttt{async\{\}} block translation.]
CODE FUN:
var tmp int = async{ return increment(5) };

CODE F#:
let mutable _task_tmp = Async.StartAsTask( async{ return 0})
let tmp = ref (0)
let _par_<unique_index> : int  = 5
_task_tmp <- Async.StartAsTask( async{
	return increment _par_<unique_index>
						})
\end{lstlisting}

For what concerns the dasync{} conversion, a few functions have been designed and inserted in the \textit{funwaputility.dll} library, explained in the following section.

\section{Funwap Utility}
The \textit{funwaputility.dll} provides \fsharp runtime support for the compiled code. In particular for the \texttt{readln} function and the \textit{dasync}. The readln support consists of a while loop that handles the exceptions raised up until the input is not a number.\\

\begin{lstlisting}[caption=All \fwap \texttt{readln()}'s are compiled into \fsharps \texttt{\_readln()}'s.]
let _readln() =
        let mutable tmp = true
        let input = ref(0)
        while tmp do
            try
                input := Convert.ToInt32(Console.ReadLine())
                tmp <- false;
            with
            | :? System.FormatException as ex ->
                Console.WriteLine("funW@P->F# Insert input again:")
        !input
\end{lstlisting}

On the other side the dasync is provided with these two methods:
\begin{itemize}
	\item \texttt{getPostAsyncInt (url:string, data)}
	\item \texttt{getPostAsyncBool (url:string, data)}
\end{itemize}
that %%%describe what they do...
 


