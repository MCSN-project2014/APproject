\documentclass[]{final_report}
\usepackage{graphicx}
\usepackage{color} 
\usepackage{hyperref} %must be upload inthe last line.

%%%%%%%%%%%%%%%%%%%%%%
%%% Input project details
\def\studentname{Technical Report}
\def\projecttitle{An Imperative-Functional Programming Language}
\def\supervisorname{Advanced Programming (301AA)}
\def\moderatorname{Gianluigi Ferrari - Antonio Cisternino}

\lstset{
basicstyle=\note\ttfamily,%
keywordstyle=\color{blue},%\bfseries,%
commentstyle=\color{green},%
stringstyle=\color{red},%
numberstyle=\tiny,%
captionpos=b,
stepnumber=5,%
numbers=left,%
language=C++,
frame=single
}

\newcommand{\fwap}{\emph{funW@P} }
\newcommand{\fsharp}{F\texttt{\#} }
\begin{document}



\maketitle
\tableofcontents\pdfbookmark[0]{Table of Contents}{toc}\newpage

%%%%%%%%%%%%%%%%%%%%%%
%%% Your Abstract here

\begin{abstract}

\begin{flushright}
\textsl{``A man provided with paper, pencil, and rubber, and subject to strict discipline, is in effect a universal machine.''\\}
[A. Turing]
\end{flushright}

The aim of this report is to offer an overview of the  \fwap  implementation and of the main design choices, made during its development. Also, it is intended both as a synthetic reference and a manual for the user wishing to use or extend our simple language.

The report structure follows closely the project's one and focuses on the most important developing phases. After reporting the complete description of the \fwap grammar and a brief summary about the tools used for generating the tokenizer, the parser and the type-checker (see Chapter~\ref{chapter2}), a recapitulation about the chosen intermediate representation (see Chapter~\ref{chapter3}) is offered. Then, the language interpreter (see Chapter~\ref{chapter4}) and the \fsharp compiler (see Chapter~\ref{chapter5}) are reviewed. To conclude, Chapter~\ref{chapter6} and ~\ref{chapter7} are an overview of the testing phase and a quick user guide respectively.

Since an actual printed document may seem an old-fashioned tool to the modern reader, all the information provided here will be also available at the official Wiki of the project on \href{https://github.com/MCSN-project2014/APproject}{GitHub}\cite{proj}. 

\end{abstract}

\newpage

%%%%%%%%%%%%%%%%%%%%% Introduction

\chapter{Introduction}

\fwap is a modern domain specific programming language (DSL) that has an imperative structure, with parallel execution support and higher order functions \cite{exercise}. 

The developing team working at project was structured as an Agile one, with three collaborating, self-organizing teams working to the implementation of:
\begin{enumerate}
	\item the parser/tokenizer module with Coco/R
	\item the interpreter, written in C\texttt{#}
	\item the \fsharp compiler, also written in C\texttt{#}
\end{enumerate}
respectively. 

Eventually, the whole group worked together to the server performing the distributed tasks and to the online version of the compiler and the interpreter, available on GitHub \cite{proj}.


\chapter{\label{chapter2} Designing the Language}

The grammar design has been a complex, delicate and continuous task of the project, since it represents the true definition of a language in formal terms and it revealed to be of crucial importance for all the subsequent phases. Lots of on the fly adjustments and corrections leaded to the definition reported and discussed in the following section.

\section{BNF Grammar}

The whole grammar is described by means of the Backus-Naur Form (BNF, \cite{bnf}), despite the fact that the Coco/R metasyntax is much closer to the Extended Backus-Naur Form (EBNF, \cite{ebnf}), since the authors retained it less readable.  We first introduce the used tokens, which are the following ones:

\begin{lstlisting} [caption=The \fwap tokens.]
TOKENS
ident  = letter {letter | digit}.
url = ap "http://" {UrlInLine} ap.
number = digit {digit}.
string = '"' {AnyButDoubleQuote | "\\\""} '"'.
\end{lstlisting}

Then, we provide all the needed productions. Even though the grammar is not LL(1), the Coco/R parser generator was able to handle LL(k) situations.

\begin{grammar}
<Fun> ::=  { <ProcDecl> } 

<ProcDecl> ::=  
'fun' <Ident> '(' <FormalsDeclList> ')' <FRType> '\{' <Bolck> '\}'
\alt 'fun' 'main' '(' ')' '\{' <Bolck> '\}' 

<AProcDecl> ::= 'fun' '(' <FormalsDeclList> ')' <FRType> '\{' <Bolck> '\}'

<Block> ::=  <VarDecl> <Block> | <Stat> <Block> | $\varepsilon$

<FormalsDeclList> ::= 	<FormalsDeclTail> 
						\alt <Ident> <Type> <FormalsDeclTail> 

<FormalsDeclTail> ::= ',' <Ident> <Type> | $\varepsilon$

<Stat> ::= 
			<Ident> '=' 'async' '\{' 'return' <Ident> '(' <ActualSyncList> ')' '\}' ';'
			\alt <Ident> '=' 'dasync' '\{' <Ident> ',' 'return' <Ident> '(' <ActualSyncList> ')' '\}'';'
			\alt <Ident> '=' 'dasync' '\{' <URL> ',' 'return' <Ident> '(' <ActualSyncList> ')' '\}'';'
			\alt <Ident> '=' <CompleteExpr> ';'
			\alt <Ident> '=' <AProcDecl>                                  
			\alt <Ident> '=' 'readln' '(' ')'';'
			\alt 'if' <CompleteExpr> '\{' <Bolck> '\}' <Else>                                        
			\alt 'while' <CompleteExpr> '\{' <Bolck> '\}'                                                      
			\alt 'println' '(' <CompleteExpr> ')' ';'
			\alt 'println' '(' <String> ')' ';'
			\alt <Return>.

<Else> :: =  'else' '\{' <Bolck> '\}' | $\varepsilon$

<Return> ::= 
			'return' <CompleteExpr>
			\alt 'return' <AProcDecl>

<VarDecl> ::= 	<VarDeclList> ';' 
				'var' <Ident>  <Type> '=' 'readln' '('')'';'
				\alt 'var' <Ident>  <Type> '=' <CompleteExpr> ';' 
				\alt 'var' <Ident>  <Type> '=' <AProcDecl> ';'
				\alt 'var' <Ident>  <Type> '=' <URL> ';'
				\alt 'var' <Ident>  <Type> '=' 'async' '\{' 'return' <Ident> '(' <ActualSyncList> ')''\}' ';'
				\alt 'var' <Ident>  <Type> '=' 'dasync' '\{'<Ident> ',' 'return' <Ident> '(' <ActualSyncList> ')' '\}' ';'
				\alt 'var' <Ident>  <Type> '=' 'dasync' '\{'<URL> ',' 'return' <Ident> '(' <ActualSyncList> ')' '\}' ';'
							
<VarDeclList> ::= 'var' <Ident> <VarDeclTail>

<VarDeclTail> ::= <Type> | ',' <Ident> <VarDeclTail> | $\varepsilon$ 

<ActualSyncList> ::= <ActualSyncTail>
\alt <CompleteExpr> <ActualSyncTail>

<ActualSyncTail> ::= ',' <CompleteExpr> | $\varepsilon$

<CompleteExpr> ::= <Expr> <CompleteExprTail> 

<CompleteExprTail> ::= <BoolOp> <Expr> <CompleteExprTail> | $\varepsilon$
		  
<Expr> ::= <SimpExpr> <ExprTail>

<ExprTail> ::=  <RelOp> <SimpExpr> <ExprTail> | $\varepsilon$
	  
<SimpExpr> ::= <Term> <SimpExprTail>

<SimpExprTail> ::=  <AddOp> <Term> <SimpExprTail> | $\varepsilon$

<Term> ::= <Factor> <TermTail>

<TermTail> ::= <MulOp> <Factor> <TermTail> | $\varepsilon$

<Factor> ::= 
		<Ident>
		\alt <Ident> '(' <ActualsList> ')' 
		\alt 'number' 
		\alt '-'<Factor> 	
		\alt 'true' 
		\alt 'false' 
		\alt '(' <CompleteExpr> ')'
		
<ActualsList> ::= 	<ActualsListTail> 
					\alt <CompleteExpr> <ActualsListTail>
					\alt <AProcDecl> <ActualsListTail>

<ActualsListTail> ::= 	',' <CompleteExpr> <ActualsListTail>
						\alt ',' <AProcDecl> <ActualsListTail>
						\alt $\varepsilon$					
		
<FRType> ::= 'fun' '(' <TypeList> ')' <FRType>
			\alt 'int' 
			\alt 'bool'

<TypeList> ::= 	<TypeListTail> 
				\alt <Type> <TypeListTail>

<TypeListTail> ::= ',' <Type> <TypeListTail> | $\varepsilon$	
			
<Type> ::= 'fun' | 'int' | 'bool' | 'url'

<AddOp> ::= '+' | '-'
	  
<RelOp> ::= '<' | '>' | '==' | '!=' | '$\leq$' | '$\geq$'
		  
<BoolOp> ::= '$\& \&$ ' | '$\|$ '

<MulOp> ::= '*' | '/'

<Ident> ::= ident

<URL> ::= url	

\end{grammar}

The Coco/R input, provided with all the semantic annotations, is contained within the file \textit{GramWithSemantics.ATG}.

\section{Symbol Table and Type Checking}

The \texttt{SymTable.cs} class represents the \fwap symbol table which - being the scope static by design choice - allows both type and environment checking. In fact, a \texttt{SymTable.cs} can register the name of any variable and declared function with all their type information (i.e. also the return type for the functions). In order to manage those associations, the \texttt{Obj} class has been coded; an \texttt{Obj.cs} instance can be

\begin{itemize}
	\item a variable
	\item a declared function
	\item a scope
\end{itemize}

distinguished by the \texttt{kind} field. All possible kinds are listed below.

\begin{lstlisting}[caption=Labels for \texttt{Node}'s.]
public enum Kinds {var,fundec,scope}
\end{lstlisting}

\texttt{Obj.cs}'s are then inserted as terminal within the Abstract Syntax Tree (AST, see Chapter\ref{chapter3}) that allows the interpreter and the compiler to access the type information of each variable. Type checking is executed while parsing the \fwap code, accordingly to the semantics rules (i.e. pieces of code) defined within the \textit{GramWithSemantics.ATG} file. In particular
\begin{itemize}
	\item the variable declarations/assignments (e.g. \texttt{var x bool = 35;} is not valid)
	\item the formal vs. actual parameters for a function call of a declared procedure
	\item the return type of declared functions 
\end{itemize}
are statically checked. On the other hand, because of the language syntax, it was not possible to type-check nor the return type of the anonymous functions nor the actual parameters coherence with respect to the formal ones, after the assignment of the function itself to a \texttt{fun} variable.

  		% explains the design of the language


\chapter{\label{chapter3} The Abstract Syntax Tree (AST)}

			% expalins the AST
\chapter{\label{chapter4} The \fwap Interpreter}

The interpreter module is capable of running the \fwap intermediate representation (AST) and produce the correct results. The executable for this piece of software is contained within the release file \textit{funwapi.exe}. 

\section{\texttt{Environment.cs}}

Each instance of \texttt{Environment.cs} keeps track of the value of the variables in a given function (main included), during the execution of a program. It is then of fundamental importance for the interpreter to work. It is structured as a \texttt{List<Dictionary<string, object>>} because it must maintain the associations related to each single scope, i.e. when the code enters/exits a scope the a dictionary is added/removed from the list.

The method \texttt{addValue(Obj var, object value)} inserts a new variable (and the associated value) within the last opened scope; the method \texttt{updateValue(Obj var, object value)} search for the first variable with a given name in the 'nearest' scope and updates its value (according to the static scope rule).

  % expalins the interpreter

\chapter{\label{chapter5} The \fsharp Code Generator}

The code generator module is responsible for translating the \fwap intermediate representation (AST) into compilable \fsharp programs. The module itself can be thought of as a fully working \textit{compiler} (i.e. a computer program performing a translation from a source into another target language, \cite{dragon}). The executable for this piece of software is contained within the release file \textit{funwapc.exe}. 

During this chapter, we will go through the organization of the \texttt{FSCodeGen.cs} class, in which the whole translator has been implemented.

\section{The \texttt{FSCodeGen.cs} class}

The overall design of the compiler could be intuitively explained by having a look at the method \texttt{translate (ASTNode n)}, partially listed below. \\

\lstinputlisting[firstline=70,lastline=118,caption={The \texttt{translate(ASTNode n)} method.}]{../APproject/FSCodeGenerator/FSCodeGen.cs }

Depending on the node's label, \texttt{translate(ASTNode n)} calls the appropriate method to perform the translation and it is recursively called by the auxiliary method \texttt{translateRecursive (ASTNode n)}, which also prints the terminal nodes when needed (i.e. the AST leaves). The aforementioned process performs a Breadth-First search over the AST structure and writes the \fsharp code into a specified output file (\textit{a.fs} by default).

While a good amount of the translating methods are trivial to understand for any programmer, a more detailed treatment is reserved to the assignments translation methods, \texttt{translateAssig(ASTNode n)} and \texttt{translateAssigDecl(ASTNode n)}, which also compiles the \textbf{async\{\}} and \textbf{dasync\{\}} constructs.

\section{Translating the Assignments}

The assignment translation -- and, analogously, the declaration/assignment -- was subject to a subtle work in order to manage the anonymous functions and the \textbf{async\{\}} and \textbf{dasync\{\}} conversion into \fsharp. 

Firstly, using the \texttt{ref} cell variables everywhere in the target code was an inducted design choice, since \fsharp anonymous functions cannot modify external \texttt{mutable} variables. Even if \texttt{mutable}'s were the team's first choice, being lighter for the target run-time support, the mixed use with \texttt{ref} was not worth the implementation effort. Also, it turned out to be fairly simple to add the \textbf{!} (\textit{bang} operator) when it was needed on the right handside of some assignment statement.

Secondarily, the same class \texttt{Environment.cs} used for the interpreter (see Chapter~\ref{chapter4}) has been adopted here in order to keep track of the association between each variable and a boolean value, set to true when that given variable is object of an async/dasync operation.The boolean value is then reset to false, immediately before issuing another non-async/non-dsync assignment of the variable. Asynchronous calls are translate into .NET \texttt{Task}'s, as shown below.

\begin{lstlisting}[caption=Example of \texttt{async\{\}} block translation.]
CODE FUN:
var tmp int = async{ return increment(5) };

CODE F#:
let mutable _task_tmp = Async.StartAsTask( async{ return 0})
let tmp = ref (0)
let _par_<unique_index> : int  = 5
_task_tmp <- Async.StartAsTask( async{
	return increment _par_<unique_index>
						})
\end{lstlisting}

		% explains the f# code generator
\chapter{\label{chapter6} The System Testing}

The code testing has been conducted during the entire development of the project. During the last weeks the system has been tested on the whole, therefore a proper set of \fwap programs has been written and collected within the folder \textsl{Code_funW@P}. 

Here is the list of the tests and what aspects of the project they are checking:
\begin{itemize}
	\item \textit{helloWorld.fun}, it is the classiv test function for any language all over the world
	\item \textit{bracket.fun}, it tests bracketed expressions and operators
\item \textit{multipleDeclarations.fun}, it attempts to declare lots of variables, all together!
	\item \textit{anonymousClosureText.fun}, it is the example for function closure reported in the text of the assignment \cite{exercise}
	\item \textit{anonymousFuncs.fun}, yet another test for closures of anonymous function which defines a tick() counter
	\item \textit{anonymousFunctionClosure.fun}, the last anonymous functions test
	\item \textit{boolReduceAsync.fun}, it asynchronously computes the maximum among five numbers according to a reduce pattern
	\item \textit{fibonacci.fun}, a test in honour of a well famous "`pisano"', computing the 6th number of the fibonacci sequence
	\item \textit{fibAsync.fun}, it asynchronously computes the first five number of the Fibonacci sequence
	\item \textit{ECTS.fun}, it is a test for nested if's but is also a useful program converting university grades from the italian system into the European Credit Transfer System \cite{ects}
\item \textit{whileAsync.fun}, it repeatedly runs four asynchronous computations
\item \textit{whileDasync.fun}, same as before but with \texttt{dasync\{\}}
\item \textit{whileAsyncDasync.fun}, mixed \texttt{dasync\{\}} and\texttt{async\{\}}, same computation as before
\item \textit{whileRead.fun}, if you input the right number, it gives you a very nerdy Answer.
\end{itemize}
  	% expalins  the testing
\chapter{\label{chapter7} A Quick User Guide}	% user guide manual 

%%%% ADD YOUR BIBLIOGRAPHY HERE

\cleardoublepage  					% begin the bibliogaphy in a new page
\phantomsection  					 % compulsory if is present {hyperref}
\addcontentsline{toc}{chapter}{\bibname} % add the bibliography in the index


\begin{thebibliography}{99}
\bibitem{DAWSON:2000} Christian Dawson. \emph{The Essence of Computing Projects -- A Student's Guide}. Pearson, 2000.
\bibitem{exercise}\emph{Advanced Programming Project 2014}. \href{http:\/\/www.di.unipi.it\/\~giangi\/CORSI\/AP\/LEZIONI2014\/PA-Proj.pdf}{\url{http://www.di.unipi.it/~giangi/CORSI/AP/LEZIONI2014/PA-Proj.pdf}}
\bibitem{ebnf}\emph{EBNF on Wikipedia}. \href{http:\/ \/en.wikipedia.org\/wiki\/Extended\_Backus\%E2\%80\%93Naur\_Form}{\url{http://en.wikipedia.org/wiki/Extended\_Backus\%E2\%80\%93Naur\_Form}}.
\bibitem{dragon}Lam, Monica, Ravi Sethi, J. D. Ullman and Alfred Aho. \emph{Compilers: Principles, Techniques, and Tools}. Addison-Wesley, 2006.
\bibitem{proj}\emph{Official Repository of the Project}. \href{https:\/\/github.com\/MCSN-project2014\/APproject}{\url{https://github.com/MCSN-project2014/APproject}}.

\end{thebibliography}



\end{document}

\end{article}
