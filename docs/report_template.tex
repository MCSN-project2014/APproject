\documentclass[]{final_report}
\usepackage{graphicx}
\usepackage{hyperref} %must be upload inthe last line.

%%%%%%%%%%%%%%%%%%%%%%
%%% Input project details
\def\studentname{Technical Report}
\def\projecttitle{An Imperative-Functional Programming Language}
\def\supervisorname{Advanced Programming (301AA)}
\def\moderatorname{Gianluigi Ferrari - Antonio Cisternino}


\newcommand{\fwap}{\emph{funW@P} }
\newcommand{\fsharp}{F\texttt{\#} }
\begin{document}

\maketitle
\tableofcontents\pdfbookmark[0]{Table of Contents}{toc}\newpage

%%%%%%%%%%%%%%%%%%%%%%
%%% Your Abstract here

\begin{abstract}

\begin{flushright}
\textsl{``A man provided with paper, pencil, and rubber, and subject to strict discipline, is in effect a universal machine.''\\}
[A. Turing]
\end{flushright}
The aim of this report is to offer an overview of the  \fwap  implementation and of the main design choices, made during its development. Also, it is intended both as a synthetic reference and a manual for the user wishing to use or extend our simple language.

Since an actual printed document may seem an old-fashioned tool to the modern reader, all the information provided here will be also available at the official Wiki of the project on \href{https://github.com/MCSN-project2014/APproject}{GitHub}\cite{proj}. 

\end{abstract}

\newpage

%%%%%%%%%%%%%%%%%%%%% Introduction

\chapter{Introduction}

The report structure follows closely the project's one and focuses on the most important developing phases. After reporting the complete description of the \fwap grammar and a brief summary about the tools used for generating the tokenizer, the parser and the type-checker (see Chapter~\ref{chapter2}), a recapitulation about the chosen intermediate representation (see Chapter~\ref{chapter3}) is offered. Then, the language interpreter (see Chapter~\ref{chapter4}) and the \fsharp compiler (see Chapter~\ref{chapter5}) are reviewed. To conclude, Chapter~\ref{chapter6} and ~\ref{chapter7} are an overview of the testing phase and a quick user guide respectively.

\chapter{\label{chapter2} Designing the Language}

The grammar design has been a complex, delicate and continuous task of the project, since the grammar represents the true definition of a language in formal terms and it revealed to be of crucial importance for all the subsequent phases. Lots of on the fly adjustments and corrections leaded to the definition reported in the following section.

\section{EBNF Grammar}

The whole grammar is described by means of the Extended Backus-Naur Form (EBNF, \cite{ebnf}) which is metasyntax, commonly adopted by parser generators nowadays. We first introduce the used tokens, which are the following ones:

\begin{lstlisting}
TOKENS
ident  = letter {letter | digit}.
url = ap "http://" {UrlInLine} ap.
number = digit {digit}.
string = '"' {AnyButDoubleQuote | "\\\""} '"'.
\end{lstlisting}

Then we provide all the needed productions:

\section{Submission of Project Report}  		% explains the design of the language
\chapter{\label{chapter3} The Abstract Syntax Tree (AST)}

The abstract class \texttt{ASTNode.cs} represents the building block of the data structure that is used as the Intermediate Representation (IR) for the \fwap language. The Abstract Syntax Tree (AST), assembled using the methods provided by the \texttt{ASTGenerator.cs} class, is the input for either the interpreter (see Chapter~\ref{chapter4}) and the \fsharp code generator (see Chapter~\ref{chapter5}). As said before, it is up to the parser, opportunely guided by the semantics annotations of Coco/R, to create the AST related to a certain \fwap source file.

\section{\texttt{Node}'s and \texttt{Term}'s}

The subclasses of \texttt{ASTNode.cs} act as a single node of the AST of a program and can be:
\begin{itemize}
	\item a \texttt{Node}, meaning an internal node of the tree 
	\item a \texttt{Term}, meaning a leaf of the tree.
\end{itemize}

Each and every \texttt{Node} is distinguished by a label which describes its function and has possibly got a list of children nodes (\texttt{List<ASTNode> children}). All the allowed labels are listed below.\\

\begin{stlisting}[caption=Labels for \texttt{Node}s.]
public enum Labels {Program, Main, Afun, FunDecl, For,
If, While, Block, Assig, Decl, AssigDecl, Return, 
Async, Print, Read, Dsync, Plus, Mul, Minus, Div, Gt,
Gte, Lt, Lte, Eq, NotEq, And, Or, Negativ, Bracket, 
FunCall};
\end{lstlisting}

\texttt{Term}'s contain a generic \texttt{object} that could be one among an integer, a boolean, a string or an \texttt{Obj} that acts as a variable (see Section\ref{typecheck}).

\section{Building the AST}

The following pictures illustrate how some labeled \texttt{Node}'s are built by the parser. An agreement on the order of the children from left to right was inevitable for implementing the interpreter and the \fsharp compiler.

\newpage
\begin{figure}
	\centering
		\includegraphics[width=0.90\textwidth]{C:/Users/Stefano/Documents/GitHub/APproject/docs/nodesAST/Program.jpg}
	\caption{Order of the children of some labeled \texttt{Node}'s; curly braces indicates zero or n copies of a certain node, round braces a choice among a few options.}
	\label{fig:Program}
\end{figure}


			% expalins the AST

\chapter{\label{chapter4} The \fsharp Code Generator}

The compiler that translate a program written in \fwap into a \fsharp syntax. 
The compiler is the \textit{funwapc} program. \href{https://github.com/MCSN-project2014/APproject/wiki/How-to-use-the-Compiler}{Here} you can find a  wiki to  use it.

\section{The \textit{FSCodeGen} class}
The \textit{FSCodeGen.cs} class is the main class performing the translation.
Starting from the AST (Abstract syntax tree) of the program tranlsate recursivly each node of the tree in a \fsharp code.


		% explains the f# code generator
\chapter{\label{chapter4} The \fwap Interpreter}

The interpreter module is capable of running the \fwap intermediate representation (AST) and produce the correct results. The executable for this piece of software is contained within the release file \textit{funwapi.exe}. 

\section{\texttt{Environment.cs}}

Each instance of \texttt{Environment.cs} keeps track of the value of the variables in a given function (main included), during the execution of a program. It is then of fundamental importance for the interpreter to work. It is structured as a \texttt{List<Dictionary<string, object>>} because it must maintain the associations related to each single scope, i.e. when the code enters/exits a scope the a dictionary is added/removed from the list.

The method \texttt{addValue(Obj var, object value)} inserts a new variable (and the associated value) within the last opened scope; the method \texttt{updateValue(Obj var, object value)} search for the first variable with a given name in the 'nearest' scope and updates its value (according to the static scope rule). Also a method to get the value of a variable is provided.

\section{Interpreting \fwap}

The interpreter starts with the method \texttt{Start()} which stores the name of the declared functions and a pointer to them and calls the \texttt{Interpret(ASTNode n, Environment env)} over the main node (creating a new \texttt{Environment.cs}). 

While the \texttt{Interpret(ASTNode n, Environment env)} method is able to execute all the statements (\texttt{while}, \texttt{readln}...), the \texttt{InterpretExp(ASTNode n, Environment env)} is called to interpret any boolean or arithmetic expression. Recursive calls to both of them allow to follow the code flow and produce results.

A special case of interpretation 
  % expalins the interpreter
\chapter{\label{chapter6} The System Testing}

Testing of each module has been carried on while programming them.   	% expalins  the testing
\chapter{\label{chapter7} A Quick User Guide}	% user guide manual 

%%%% ADD YOUR BIBLIOGRAPHY HERE

\cleardoublepage  					% begin the bibliogaphy in a new page
\phantomsection  					 % compulsory if is present {hyperref}
\addcontentsline{toc}{chapter}{\bibname} % add the bibliography in the index


\begin{thebibliography}{99}
\bibitem{DAWSON:2000} Christian Dawson. \emph{The Essence of Computing Projects -- A Student's Guide}. 192 pages. ISBN: 013021972X. Pearson Education, 2000.
\bibitem{ebnf}\emph{EBNF on Wikipedia}. \href{http:\/\/en.wikipedia.org\/wiki\/Extended\_Backus\%E2\%80\%93Naur\_Form}{http:\/\/en.wikipedia.org\/wiki\/Extended\_Backus\%E2\%80\%93Naur\_Form}
\end{thebibliography}


\end{document}

\end{article}
